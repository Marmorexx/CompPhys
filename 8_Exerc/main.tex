
\documentclass{article}

\usepackage[margin=4cm]{geometry}
\usepackage{polyglossia}
	\setmainlanguage{english}
\usepackage{amsmath}
\usepackage{amssymb}
\usepackage{siunitx}
\usepackage{float}
\usepackage{booktabs}
\usepackage{subcaption}
\usepackage{graphicx}
\usepackage{xcolor}
\usepackage{listings}
    \lstset{language=Python,
	basicstyle=\footnotesize\ttfamily,
	breaklines=true,
	framextopmargin=50pt,
	frame=bottomline,
	backgroundcolor=\color{white!85!black},
	commentstyle=\color{blue},
	keywordstyle=\color{red},
	stringstyle=\color{orange!80!black}}
\usepackage{tikz}

\title{Computational Physics - Exercise 8}
\author{Maurice Donner \and Lukas Häffner}

\begin{document}

\maketitle
\newpage

\section{Fixed Points of the Lorenz dynamical System}

The Lorenz attractor problem is given by the following coupled set of 
differential equations
\begin{align}
    \dot x &= - \sigma (x - y ) \\
    \dot y &= rx - y - xz \\
    \dot z &= xy - bz
    \label{Lorenz}
\end{align}
The fixed points for this problem are \( \lambda_1 = (0,0,0) \) for all \( r \)
and \( \lambda_{2,3} = (\pm a_0, \pm a_0, r-1 )\) with \( a_0 = 
    \sqrt{b(r-1)}\) for \( r > 1 \).\\
In this exercise we want to examine the stability of \( \lambda_{2,3} \) by the
Jacobian taken at the fixed points, and then looking for its eigenvalues by means
of finding the zero points of the following characteristic polynomial:
\begin{align}
    P(\lambda) = \lambda^3 + (1+b+\sigma)\lambda^2 +b(\sigma+r)\lambda
    + 2 \sigma b (r-1)
\end{align}
We first plot $P(\lambda)$ as a function of \( \lambda \). For that we use
\( \sigma = 10, b = 8/3 \):
\begin{figure}[H]
    \centering
    \includegraphics[width=9cm]{polynomial.pdf} 
    \caption{Characteristic Polynomial of the Lorenz attractor} 
\end{figure}
To check if the points are stable, one has to examine the area around the point.
Lies the point in an upwards slope, it is unstable. Otherwise, it is stable.
In case of a saddle point, no definite answer can be made.\\
That means the only stable points we have, are the ones for values of
\( r < 1.3456 \). For larger \( r \) two of the 3 stationary points vanish.\\

Next we determine the (complex) roots for different values of r: \\

\begin{table}[H]
    \centering
\begin{tabular}{lrr}
    \toprule 
    r & Re(x) & Im(x) \\ \midrule 
    1.3456 & -11.09 & 0 \\
    1.3456 & -1.3 & 0 \\
    1.3456 & -1.28 & 0 \\ \midrule
    1.5 & -11.13 & 0 \\
    1.5 & -1.27 & 0.88 \\
    1.5 & -1.27 & -0.88 \\ \midrule
    24 & -13.62 & 0 \\
    24 & -0.02 & 9.49 \\
    24 & -0.02 & -9.49\\ \midrule
    28 & -13.85 & 0 \\ 
    28 & 0.09 & 10.19 \\
    28 & 0.09 & -10.19 \\
    \bottomrule 
\end{tabular}
\caption{(Complex) roots of the Characteristic Polynomial} 
\end{table}
For \( r < r_\text{crit,1} = 1.3456 \), all Fixpoints are \( < 0 \) and real,
and are hence stable.\\
For \(  r_\text{crit,1} < r < r_\text{crit,2} = 24.74\), the real parts of the
fixed points are still \( < 0 \) and the Fixpoints (including the complex
conjugate ones) are hence stable. \\
For \( r > r_\text{crit,2} \), the real part of the complex conjugate solutions
becomes \( > 0 \). The Fixed points are hence unstable. \\

\section{The Lorenz attractor}
We solve the Lorenz equations numerically with \texttt{rk4}, for the values
\( r = 0.5, 1.15, 1.3456, 24 \ \text{and} \ 28 \).
For that we use the previously integrated Runge-Kutta-4 algorithm
(see Exercise 2). All we need to do is to integrate the coupled Lorenz equations:
\begin{lstlisting}
def f(y0,x0): # y0 array that consists of [x,y,z]
    deriv = np.array([
    - sig*(y0[0] - y0[1]),
    r*y0[0] - y0[1] - y0[0]*y0[2],
    y0[0]*y0[1] - b*y0[2]])

    return deriv
\end{lstlisting}
Using \texttt{rk4}, we plot the trajectories for all \( r \) each with Starting
point \( C_+ \ \text{and} \ C_- \):
\begin{figure}[H]
    \centering
    \includegraphics[width=\textwidth]{Figure2-1.pdf} 
    \caption{Determining Fixed points of the Lorenz dynamical System} 
\end{figure}
As discussed before, the fixed points for 
\end{document}

