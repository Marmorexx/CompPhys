
\documentclass{article}

\usepackage{polyglossia}
	\setmainlanguage{german}
\usepackage{amsmath}
\usepackage{siunitx}
\usepackage[margin=4cm]{geometry}
\usepackage{booktabs}
\usepackage{graphicx}
\usepackage{amssymb}

\title{Introduction to Computational Physics}
\author{Prof. Ralf Klessen \& Prof. Rainer Spurzem}

\begin{document}

\maketitle
\newpage

% {{{ 1 - Administration
\section{Administration}

Recieve Homework assignment on wednesday, hand in next week's Friday.
Tutorials will start 26.04.2019

Tutor Email: yun@mpia.de

220 Points - 60\% required on the exercise Sheets

\textit{The art of scientific Computing} (Literature recommendation) \\
% }}}

% {{{ 2 - Week 1
This lecture is all about making good choices. For example we can approximate
\( \pi \) by creating a geometric shape and counting the circumference while
adding corners. For the algortihm we should use only additions, no substractions
, since that will midigate rounding errors better. 

\section*{Ordinary Differential Equations} 
\begin{enumerate}
    \item Example: \textbf{Gravitational 2-Body Problem}\\
	Look at relative Position between two bodies
	\[ 
	    \vec r = \vec r_1 - \vec r_2
	\]
	Newton:
	\[ 
	    \vec \ddot r = \frac{\partial ^2 \bar r}{\partial t ^2} =
	    -G \frac{M}{r^2} \frac{\bar r}{r}
	\]
	\[ 
	    r = | \bar r | ; G = 6.67 \cdot 10^{-8} \si{cm^3/gs^2} 
	\]
	Force 2 on 1:
	\[ 
	    \bar F _{12} = m _{1} \bar a _{1}
	\]
	Force 1 on 2:
	\[ 
	    \bar F _{21} = m _{2} \bar a _{2} 
	\]
	Newton:
	\[ 
	    \bar F _{21} = - \bar F _{12} 
	\rightarrow \bar a _2 = - \frac{m_1}{m_2} \bar a_1
	\]
	\[ 
	    \bar a = \bar a_1 - \bar a_2 = \bar a_1 + \frac{m_1}{m_2} \bar a_1
	    = \frac{m_1 + m _2}{m_1 m_2} m_1 \bar a_1 = \frac{m_1 + m _2}{m_1 m_2}
	    \bar F_{12}
	\]
	\[ 
	    \mu = \frac{m_1 + m _2}{m_1 m_2}
	\]
	\[ 
	    \mu  \ddot \bar r = -G \frac{m_1 m_2}{r^2} \frac{\bar r}{r}
	\]
	How t simplify?
	\[ 
	    \bar \dot r = \bar u
	\]
	\[ 
	    \bar \dot u = - \frac{GM}{r^2} \frac{r^2}{r}
	\]
	ODE of order N \( \rightarrow \) system of 1st order ODEs
	
	
	

\end{enumerate}
% }}}

\end{document}

